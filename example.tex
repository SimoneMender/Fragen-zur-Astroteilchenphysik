\PassOptionsToPackage{unicode}{hyperref}
\documentclass[aspectratio=1610, professionalfonts, 9pt]{beamer}

\usefonttheme[onlymath]{serif}
\usetheme[showtotalframes]{tudo}

\ifluatex
  \usepackage{polyglossia}
  \setmainlanguage{german}
\else
  \ifxetex
    \usepackage{polyglossia}
    \setmainlanguage{german}
  \else
    \usepackage[german]{babel}
  \fi
\fi


% Mathematik
\usepackage{amsmath}
\usepackage{amssymb}
\usepackage{mathtools}
\usepackage{cancel}

\usepackage{hyperref}
\usepackage{bookmark}

%%%%%%%%%%%%%%%%%%%%%%%%%%%%%%%%%%%%%%%%%%%%%%%%%%%%%%%%%%%%%%%%%%%%%%%%%%%%%%%%
%%%%%-------------Hier Titel/Autor/Grafik/Lehrstuhl eintragen--------------%%%%%
%%%%%%%%%%%%%%%%%%%%%%%%%%%%%%%%%%%%%%%%%%%%%%%%%%%%%%%%%%%%%%%%%%%%%%%%%%%%%%%%

%Titel:
\title{Fragen zur Astroteilchenphysik}
%Autor
\author[]{}
%Lehrstuhl/Fakultät
\institute[]{}
%Titelgrafik
% \titlegraphic{\includegraphics[width=\textwidth]{}
\date{16 Juli, 2018}

\begin{document}

\maketitle




%%%%%%%%%%%%%%%%%%%%%%%%%%%%%%%%%%%%%%%%%%%%%%%%%%%%%%%%%%%%%%%%%%%%%%%%%%%%%%%%
\begin{frame}{Wodurch und wie entstehen Gamma Ray Bursts?}
  \begin{columns}
 \begin{column}{0.4\textwidth}
  \begin{itemize}
    \item  Man weiß es nicht
    \item  Bei manchen Gamma Ray Bursts: Supernova
  \end{itemize}
\vspace{2em}
\end{column}
\begin{column}{0.6\textwidth}
  evtl hier noch ein Bild
% \begin{figure}
%   \centering
%   \includegraphics[width=\textwidth]{Bilder/beispiel.pdf}
% \end{figure}
\end{column}
  \end{columns}
\end{frame}

\begin{frame}


Liegt es an der Zusammensetzung der Erdatmosphäre, dass nicht alle Frequenzen durchgelassen werden? Und wie kommt es, dass bestimmte Frequenzbereiche nicht durchgelassen werden?

Wie kann interstellare Materie genau nachgewiesen werden bzw. wie funktioniert der Nachweis?

Warum haben die Sonne/ die Sterne ein Schwarzkörperspektrum?

Wodurch genau ist die Rotverschiebung charakterisiert?

Was genau ist das Wasserstoffbrennen und wodurch entsteht es?

Wie genau funktioniert der Energietransport bei Sternen? Warum leben leichte Sterne länger als schwere? Was genau ist das Eddington-Limit?   Wie sieht der Lebeneslauf eines Sternes aus?
Aus welchen Elementen bestehen die meisten Sterne?  Was sind typische Ursachen für Dichtefluktuationen, die zur Entstehung neuer Sterne führt?

Worin liegt genau der Unterschied zwischen einer Supernovae Typ I und Supernovae Typ II?
Was genau ist eine Wind-Supernovae und warum hat sie einen anderen Fluss als die SN-ISM? Wofür steht ISM?

Wie kommt es zur variablen Helligkeit der Cepheiden?

Was genau ist das Hubbelgesetz und die Hubbelkonstante und wofür ist es?

Wofür genau ist die Friedmann-Gleichung?

Was sind Feld-Quanten?

\end{frame}

\begin{frame}{Motivation}
  Was sind kosmische Phasenübergänge?

  Was sind typische Ursachen für Dichtefluktuationen, die zur Entstehung neuer Sterne führt?

  Wie sieht der Lebeneslauf eines Sternes aus?

  Aus welchen Elementen bestehen die meisten Sterne?

  Wie lässt sich der Ursprung der kosmischen Hintergrundstrahlung verstehen?

  Wiederholt sich das Universum immer wieder?

  Wie stellt man sich Dunkle Materie vor?

  Gibt es Antiteilchen für Eichbosonen?

  Gibt es Wurmlöcher? Ist deren Existenz vorstellbar, wenn ja, wie funktioniert dies?

  Wie genau kann man sich ein flaches bzw. gekrümmtes Universum vorstellen?

  Wie schafft es ein Neutronenstern den Fermidruck zu überwinden?

  Wie ist der Mikrowellenhintergrund zu erklären, wo entsteht die Strahlung
  und wie kann sie die Erde erreichen, wenn sich alles vom Zentrum des Urknalls
  wegbewegt?

  Wieso gibt es für die Energiefreigabe im Universum (z.B. durch
  Kollision von schwarzen Löchern) eine Obergrenze? Es ist doch genügend
  Energie vorhanden, wie können unterschiedliche Ereignisse davon etwas
  mitbekommen?

  Warum zeigt der Jet eines schwarzen Loches nach "oben"?
  Die Teilchen, die Drehimpuls durch andere Teilchen erhalten haben könnten
  sich doch auch in der Ebene von dem schwarzen Loch entfernen, warum nach
  "oben"?

  Woher stammt die Gravitation?

  Wie setzt sich die Masse der Elementarteilchen zusammen?

  Was passiert in der Nähe eines schwarzen Lochs?

  Bei der Untersuchung von CR ist ja ein Ziel die Lorentzverletzung zu
  zeigen, aber wie will man das tun? Wonach sucht man?

  In der Nähe eines schwarzen Loches wirken ja starke Magnetfelder. Aber
  wie verhalten sich externe Magnetfelder anderer Quellen in der Nähe des
  schwarzen Loches? Folgen die Feldlinien dem gekrümmten Raum oder inwiefern
  werden sie verzerrt.

  Wie bekommt man es hin, dass moderne Teleskope eine so gute Auflösung
  besitzen (zb. Hubble Teleskop :0,05")?
\end{frame}

\begin{frame}
  Wodurch genau entstehen kosmische Magnetfelder und wie sind sie messbar?

  Was genau ist Plasma

  Was zeichnet einen unipolaren Induktor und was eine magnetische Rekonnektion aus?

  Was genau ist der Fermimechanismus 1. und 2. Ordnung?

  Wie kann erkannt werden, wodurch Cherenkovstrahlung entsteht und welche ist von Interesse?
  Wie genau funktioniert die Detektion von Cherenkovstahlung mittels Teleskopen?

  Woher weiß man, ob die Gammastrahlung kosmischen Ursprungs ist?

  Was genau sagt die Sensitivität aus?
  Wie entstehen Neutrinoflüsse der AGNs?

  Was ist der Lyman-α-Wald?

  Was/Wofür ist das Fireball-Modell?

  Wie und wo entstehen Gravitationswellen?

  Was genau zeichnet den Fluss kosmischer Strahlung aus?
  Bei der Untersuchung von CR ist ja ein Ziel die Lorentzverletzung zu
  zeigen, aber wie will man das tun? Wonach sucht man?

  Wofür steht EBL?

  Wie ist der Mikrowellenhintergrund zu erklären, wo entsteht die Strahlung
  und wie kann sie die Erde erreichen, wenn sich alles vom Zentrum des Urknalls
  wegbewegt?

    Wie lässt sich der Ursprung der kosmischen Hintergrundstrahlung verstehen?

    Wodurch genau kommt die 3K Hintergrundstrahlung zustande? Wie genau wurde die Struktur des kosmischen Mikrowellenhintergrunds bestimmt?
    Und warum/wofür wird das zugehörige Spektrum untersucht?

  Woher kommen die unterschiedlich geschätzten Lebensdauern von Protonen?

  Wie spektroskopiert man Das Licht von astronomischen Quellen? Mit einem Prisma, oder wird das irgendwie gesamplet und digitalisiert?

  Wie funktioniert die Sternstromparallaxe?

  Wie bekommt man es hin, dass moderne Teleskope eine so gute Auflösung
  besitzen (zb. Hubble Teleskop :0,05")?

\end{frame}


\end{document}
