\begin{frame}{Motivation}
  Was sind kosmische Phasenübergänge?

  Wiederholt sich das Universum immer wieder?

  Wie stellt man sich Dunkle Materie vor?
  Wie genau funktioniert der Gravitationslinseneffekt?
  Was genau ist Dunkle Materie?

  Gibt es Antiteilchen für Eichbosonen?

  Gibt es Wurmlöcher? Ist deren Existenz vorstellbar, wenn ja, wie funktioniert dies?

  Wie genau kann man sich ein flaches bzw. gekrümmtes Universum vorstellen?

  Wie schafft es ein Neutronenstern den Fermidruck zu überwinden?

  Wieso gibt es für die Energiefreigabe im Universum (z.B. durch
  Kollision von schwarzen Löchern) eine Obergrenze? Es ist doch genügend
  Energie vorhanden, wie können unterschiedliche Ereignisse davon etwas
  mitbekommen?

  Wodurch genau sind die AGNs gekennzeichnet?
  Warum zeigt der Jet eines schwarzen Loches nach "oben"?
  Die Teilchen, die Drehimpuls durch andere Teilchen erhalten haben könnten
  sich doch auch in der Ebene von dem schwarzen Loch entfernen, warum nach
  "oben"?
  Wie entstehen die Jets der AGNs?

  Woher stammt die Gravitation?

  Wie setzt sich die Masse der Elementarteilchen zusammen?

  Was passiert in der Nähe eines schwarzen Lochs?

  In der Nähe eines schwarzen Loches wirken ja starke Magnetfelder. Aber
  wie verhalten sich externe Magnetfelder anderer Quellen in der Nähe des
  schwarzen Loches? Folgen die Feldlinien dem gekrümmten Raum oder inwiefern
  werden sie verzerrt.

\end{frame}
