\begin{frame}{Motivation}
  Was sind kosmische Phasenübergänge?

  Was sind typische Ursachen für Dichtefluktuationen, die zur Entstehung neuer Sterne führt?

  Wie sieht der Lebeneslauf eines Sternes aus?

  Aus welchen Elementen bestehen die meisten Sterne?

  Wie lässt sich der Ursprung der kosmischen Hintergrundstrahlung verstehen?

  Wiederholt sich das Universum immer wieder?

  Wie stellt man sich Dunkle Materie vor?

  Gibt es Antiteilchen für Eichbosonen?

  Gibt es Wurmlöcher? Ist deren Existenz vorstellbar, wenn ja, wie funktioniert dies?

  Wie genau kann man sich ein flaches bzw. gekrümmtes Universum vorstellen?

  Wie schafft es ein Neutronenstern den Fermidruck zu überwinden?

  Wie ist der Mikrowellenhintergrund zu erklären, wo entsteht die Strahlung
  und wie kann sie die Erde erreichen, wenn sich alles vom Zentrum des Urknalls
  wegbewegt?

  Wieso gibt es für die Energiefreigabe im Universum (z.B. durch
  Kollision von schwarzen Löchern) eine Obergrenze? Es ist doch genügend
  Energie vorhanden, wie können unterschiedliche Ereignisse davon etwas
  mitbekommen?

  Warum zeigt der Jet eines schwarzen Loches nach "oben"?
  Die Teilchen, die Drehimpuls durch andere Teilchen erhalten haben könnten
  sich doch auch in der Ebene von dem schwarzen Loch entfernen, warum nach
  "oben"?

  Woher stammt die Gravitation?

  Wie setzt sich die Masse der Elementarteilchen zusammen?

  Was passiert in der Nähe eines schwarzen Lochs?

  Bei der Untersuchung von CR ist ja ein Ziel die Lorentzverletzung zu
  zeigen, aber wie will man das tun? Wonach sucht man?

  In der Nähe eines schwarzen Loches wirken ja starke Magnetfelder. Aber
  wie verhalten sich externe Magnetfelder anderer Quellen in der Nähe des
  schwarzen Loches? Folgen die Feldlinien dem gekrümmten Raum oder inwiefern
  werden sie verzerrt.

  Wie bekommt man es hin, dass moderne Teleskope eine so gute Auflösung
  besitzen (zb. Hubble Teleskop :0,05")?
\end{frame}
