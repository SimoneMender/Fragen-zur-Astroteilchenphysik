\begin{frame}
  Wodurch genau entstehen kosmische Magnetfelder und wie sind sie messbar?

  Was genau ist Plasma

  Was zeichnet einen unipolaren Induktor und was eine magnetische Rekonnektion aus?

  Was genau ist der Fermimechanismus 1. und 2. Ordnung?

  Wie kann erkannt werden, wodurch Cherenkovstrahlung entsteht und welche ist von Interesse?
  Wie genau funktioniert die Detektion von Cherenkovstahlung mittels Teleskopen?

  Woher weiß man, ob die Gammastrahlung kosmischen Ursprungs ist?

  Was genau sagt die Sensitivität aus?
  Wie entstehen Neutrinoflüsse der AGNs?

  Was ist der Lyman-α-Wald?

  Was/Wofür ist das Fireball-Modell?

  Wie und wo entstehen Gravitationswellen?

  Was genau zeichnet den Fluss kosmischer Strahlung aus?
  Bei der Untersuchung von CR ist ja ein Ziel die Lorentzverletzung zu
  zeigen, aber wie will man das tun? Wonach sucht man?

  Wofür steht EBL?

  Wie ist der Mikrowellenhintergrund zu erklären, wo entsteht die Strahlung
  und wie kann sie die Erde erreichen, wenn sich alles vom Zentrum des Urknalls
  wegbewegt?

    Wie lässt sich der Ursprung der kosmischen Hintergrundstrahlung verstehen?

    Wodurch genau kommt die 3K Hintergrundstrahlung zustande? Wie genau wurde die Struktur des kosmischen Mikrowellenhintergrunds bestimmt?
    Und warum/wofür wird das zugehörige Spektrum untersucht?

  Woher kommen die unterschiedlich geschätzten Lebensdauern von Protonen?

  Wie spektroskopiert man Das Licht von astronomischen Quellen? Mit einem Prisma, oder wird das irgendwie gesamplet und digitalisiert?

  Wie funktioniert die Sternstromparallaxe?

  Wie bekommt man es hin, dass moderne Teleskope eine so gute Auflösung
  besitzen (zb. Hubble Teleskop :0,05")?

\end{frame}
