\begin{frame}
  Was zeichnet einen unipolaren Induktor und was eine magnetische Rekonnektion aus?

  Was genau ist der Fermimechanismus 1. und 2. Ordnung?

  Wie kann erkannt werden, wodurch Cherenkovstrahlung entsteht und welche ist von Interesse?

  Woher weiß man, ob die Gammastrahlung kosmischen Ursprungs ist?

  Was genau sagt die Sensitivität aus?
  Wie entstehen Neutrinoflüsse der AGNs?

  Wodurch genau sind die AGNs gekennzeichnet?

  Was ist der Lyman-α-Wald?

  Was/Wofür ist das Fireball-Modell?

  Wie und wo entstehen Gravitationswellen?

  Was genau zeichnet den Fluss kosmischer Strahlung aus?

  Was genau ist eine Wind-Supernovae und warum hat sie einen anderen Fluss als die SN-ISM? Wofür steht ISM?

  Wie genau funktioniert die Detektion von Cherenkovstahlung mittels Teleskopen?

  Wofür steht EBL?

  Woher kommen die unterschiedlich geschätzten Lebensdauern von Protonen?

  Wie spektroskopiert man Das Licht von astronomischen Quellen? Mit einem Prisma, oder wird das irgendwie gesamplet und digitalisiert?

  Wie funktioniert die Sternstromparallaxe?

\end{frame}
